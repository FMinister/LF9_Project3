\documentclass[a4paper,12pt,headings=small,ngerman,bibliography=totoc]{scrartcl}
\usepackage[ngerman]{babel}
\usepackage[utf8]{inputenc}
\usepackage[headsepline]{scrlayer-scrpage}

\usepackage{amssymb,graphicx}
\clearscrheadings
\pagestyle{scrheadings}
\usepackage[left=2.5cm,right=2.5cm,top=3cm,bottom=2.5cm]{geometry}
\usepackage[onehalfspacing]{setspace}
\usepackage{microtype}
\ofoot{\pagemark}
\footskip1cm
\setkomafont{sectioning}{\normalfont\normalcolor\bfseries}
\usepackage[osf,sc]{mathpazo}
\usepackage{url}
\clubpenalty = 10000
\widowpenalty = 10000 \displaywidowpenalty = 10000

\usepackage[usenames,
	    dvipsnames,
	    svgnames]{xcolor}

\usepackage[babel,german=quotes]{csquotes}
\usepackage{url}
\usepackage{soul}

%%% Für Bildunterschriften %%%
\usepackage{caption}

%%% Für Bilder im Text %%%
\usepackage{wrapfig}

%%% Für Mathe %%%
\usepackage[fleqn]{amsmath}
\usepackage{commath}
\usepackage{amstext}
\allowdisplaybreaks
%%% Für schräge Bruchstriche %%%
\usepackage{nicefrac}
%%% Rechtecke %%%
\usepackage{amsfonts}

%%% Für Textübergreifende Numerierung %%%
\usepackage{enumitem}
\renewcommand{\labelenumi}{\alph{enumi})}

%%% Für Abkürzungen %%%
\usepackage{acronym}

%%%% Überschriften bis subsubsubsection (paragraph) %%%
\setcounter{secnumdepth}{5}

\setkomafont{dictumtext}{\itshape\small}
\setkomafont{dictumauthor}{\small}
\renewcommand*\dictumwidth{\linewidth}
\renewcommand*\dictumauthorformat[1]{--- #1}
\renewcommand*\dictumrule{}

%%% Für Quotes %%%
\usepackage[ngerman]{varioref}
\usepackage{hyperref}
 \hypersetup{%draft, 								% no hyperlinking at all (useful in b/w printouts)
    colorlinks=true, breaklinks=true,
    urlcolor=Black, linkcolor=Black, citecolor=Black,
    linktoc=page, %
    bookmarksnumbered, bookmarksopenlevel=1, bookmarksdepth = section,%
    pdfstartview=FitV,
    }
\setlength{\parindent}{0em}
\usepackage{cleveref}
\crefname{paragraph}{Abschnitt}{Abschnitt}

\usepackage[bibencoding=utf8,
			sortlocale=de,
			style=authoryear-icomp,
			pagetracker=true,
			autocite=inline,
			backrefstyle=three+,
			date=short,
			sorting=nty,
			backend=biber]{biblatex}
			
\bibliography{Literaturverzeichnis}

%%% urldate in eckigen Klammern %%%
\DeclareFieldFormat{urldate}{\mkbibbrackets{#1}}
%%% URL: = Verfügbar unter: %%%
\DeclareFieldFormat{url}{{Verfügbar unter:}\space\url{#1}}
%%% Abstand zwischen den Literaturangaben %%%
\setlength{\bibitemsep}{1.3em}
%%% statt und ein & %%%
\renewcommand*{\finalnamedelim}{\space\&\space}
%%% Nachname, Vorname, immer %%%
\DeclareNameAlias{sortname}{last-first}

%%% Kopfzeile %%%
\automark[section]{section}
\renewcommand*{\headfont}{\normalfont}
\setkomafont{pageheadfoot}{}
\renewcommand*{\sectionmarkformat}{} 

%%% Damit das Abbildungsverzeichnis im Inhaltsverzeichnis abgebildet wird %%%
\usepackage{tocbibind}
\renewcommand{\listoffigures}{\begingroup
  \tocchapter
  \tocfile{\listfigurename}{lof}
  \endgroup}
  
%%% Für SI Einheiten %%%
\usepackage{siunitx}
\sisetup{
  locale = DE ,
  detect-all
}

%Für Tabellen
\usepackage{array}
% Für Seitenübergreifende Tabellen
\usepackage{longtable}
% Für Tabellen mit vertikal und horizontal zentriertem Inhalt
\newcolumntype{B}[1]{>{\centering\arraybackslash}m{#1}}

\begin{document}

\pagenumbering{gobble}
\pagestyle{empty}


\begin{center}
  \Large{Berufliches Schulzentrum für Elektrotechnik Dresden}\\
\end{center}

\begin{center}
  \Large{Fachbereich Informationstechnik}
\end{center}
\begin{verbatim}

\end{verbatim}
\begin{center}
  \textbf{\LARGE{Lernfeld 9}}
\end{center}

\begin{center}
  \Large{Segmentierung eines Enterprise IT-Netzwerkes}
\end{center}

\begin{flushleft}
  \begin{tabular}{lll}
                           &                                                            & \\
                           &                                                            & \\
                           &                                                            & \\
                           &                                                            & \\
                           &                                                            & \\
                           &                                                            & \\
    \textbf{erstellt von:} & Johannes Leyrer \flq{}i20leyrerjo@bszetdd.lernsax.de\frq{}   \\
                           &                                                            & \\
                           &                                                            & \\
    \textbf{erstellt im:}  & Lehrjahr 2, 2021/2022                                        \\
                           &                                                            & \\
                           &                                                            & \\
  \end{tabular}
\end{flushleft}

\vspace{\fill}
\begin{description}
  \item[Disclaimer:] Diese Mitschrift ist nicht offiziell. Insbesondere erhebe ich keinen Anspruch auf Vollständigkeit oder Korrektheit. Ich bin jederzeit froh um Hinweise zu Fehlern oder Unklarheiten.
\end{description}

\newpage
%%% Inhaltsverzeichnis einfügen %%%
\tableofcontents
\cleardoublepage

\newpage
\pagestyle{scrheadings}
\pagenumbering{arabic}
\ohead{Vorlage}
\ihead{\rightmark}




\end{document}
